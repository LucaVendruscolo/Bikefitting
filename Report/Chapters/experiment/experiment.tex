
% Explain the training and testing results with graphs and elaborating on why they make sense, what could be improved.
\FloatBarrier
\subsection{Training and Testing Results}
\label{subsec:results}


\begin{figure}[H]
    \centering
    
    \begin{subfigure}[c]{0.38\textwidth}
        \centering
        \includegraphics[width=\linewidth]{Chapters/experiment/IMG_3803_frame_000486.jpg}
        \caption{Sample validation frame}
        \label{fig:validation-frame}
    \end{subfigure}
    \hfill
    \begin{subfigure}[c]{0.58\textwidth}
        \centering
        \includegraphics[width=\linewidth]{Chapters/experiment/validation_line_graph.png}
        \caption{True vs predicted angle. The dip is due to wrap around.}
        \label{fig:validation-graph}
    \end{subfigure}

    \caption{Bike angle prediction validation on unseen video data.}
    \label{fig:validation-results}
\end{figure}

\FloatBarrier

\begin{table}[H]
    \centering
    \caption{Validation metrics}
    \label{tab:validation-metrics}
    \begin{tabular}{lcc}
        \toprule
        \textbf{Metric} & \textbf{Raw} & \textbf{Smoothed (30 frame window)} \\
        \midrule
        Mean Absolute Error & 10.67° & 8.87° \\
        Median Absolute Error & 8.60° & 8.26° \\
        90th Percentile Error & 23.09° & 17.15° \\
        \bottomrule
    \end{tabular}
\end{table}

\subsubsection{Crankshaft Identification Method}
