
% Include a few very relevant related works and how your work relates to those, expanding on the previous section. We do not expect you to cover all previous works.

\subsection{Related Works}
\label{subsec:related_works}


\subsubsection{Commercial bike-fit applications}

The use of software (including machine-learning-based tools) for bike fitting is well established. For example, Bike Fast Fit Elite\footnote{\url{https://apps.apple.com/cn/app/bike-fast-fit-elite/id1145619812?l=en-GB}} uses measured joint angles to recommend adjustments. A common limitation of these tools is that they assume the rider is filmed while cycling on a stationary setup (e.g., a wheel on trainer). This requirement reduces accessibility for cyclists who do not own such equipment.


\begin{samepage}
We surveyed several popular mobile applications in this category. On Android, apps with publicly visible downloads include:
\begin{itemize}
    \item PedalPro -- Fast Bike Fit: 50k+ downloads
    \item Apiir Bike Fit: 10k+ downloads
    \item Meu Bikefit+: 10+ downloads
\end{itemize}

From our review, none of these applications support bike fitting without a trainer. The closest workaround we found was suggested by Bike Fast Fit in response to the question: ``Not having a trainer, could I just lean against the wall and pedal in the reverse direction?'' They replied: ``That's a creative solution and it should work for a rough estimate. Pedaling under load changes some angles, so reverse pedaling won't be as accurate.''\cite{bikefastfit_help} While this may produce a rough estimate, it still requires the rider to remain upright while pedalling backwards, which can be unsafe or impractical in many setups.
\end{samepage}

\subsubsection{Sports Science Research}

There exists a literature on optimal bicycle mechanics and fit. Swart \cite{swart2019cycling} provides a survey of studies analyzing optimal saddle height among cyclists across different skill levels, measuring angles by various means. Peveler and Green examine the effects of seat height on anaerobic power output \cite{peveler2008effects,peveler2011effects}, specifically between the seat heights resulting from the Hamley and Holmes method of seat height calibrations. 

Other studies such as those conducted by Holiday, Fisher, Theo, and Swart \cite{holliday2017static} have attempted to estimate the differences between mechanics derived from static bike fits versus on-road realized mechanics. They reference methods from Bini, Hume, and Croft in utilizing reflective markers to measure the maximum knee flexion of a rider in a dynamic setting\cite{bini2011effects}. 