%%%%%%%%%%%%%%%%%%%%%%%%%%%%%%%%%%%%%%%%%%%%%%%%%%%%%%%%%%%%%%%%%%%%%%%%%%%%%%%%
%% NeurIPS 2023 Paper
%% Machine Learning and the Physical World - Course Project
%%%%%%%%%%%%%%%%%%%%%%%%%%%%%%%%%%%%%%%%%%%%%%%%%%%%%%%%%%%%%%%%%%%%%%%%%%%%%%%%

\documentclass{article}

% NeurIPS 2023 style (use [final] for camera-ready, [preprint] for arXiv)
\usepackage[final]{neurips_2023}
\makeatletter
\renewcommand{\And}{\end{tabular}\hfil\linebreak[0]\hfil\begin{tabular}[t]{c}}
\makeatother
% Core packages
\usepackage[numbers]{natbib}
\usepackage{graphicx}
\usepackage{xcolor}
\usepackage{hyperref}
\usepackage{url}

% Math packages
\usepackage{amsfonts}
\usepackage{amssymb}
\usepackage{bm}
\usepackage{dsfont}

% Tables and figures
\usepackage{booktabs}
\usepackage{float}
\usepackage{subcaption}
\usepackage{adjustbox}
\usepackage{placeins}

% TikZ for diagrams
\usepackage{tikz}
\usetikzlibrary{positioning,arrows.meta}

%%%%%%%%%%%%%%%%%%%%%%%%%%%%%%%%%%%%%%%%%%%%%%%%%%%%%%%%%%%%%%%%%%%%%%%%%%%%%%%%
%% Paper metadata
%%%%%%%%%%%%%%%%%%%%%%%%%%%%%%%%%%%%%%%%%%%%%%%%%%%%%%%%%%%%%%%%%%%%%%%%%%%%%%%%

\title{Bicycle Fitting From Pose Interpolation}

\author{%
  {\normalfont Luca Vendruscolo} \\
  University of Cambridge \\
  \texttt{lv373@cam.ac.uk}
  \And
  {\normalfont Kyle Fram} \\
  University of Cambridge \\
  \texttt{kdf25@cam.ac.uk}
  \And
  {\normalfont Fivos Papathanasiou} \\
  University of Cambridge \\
  \texttt{fp394@cam.ac.uk}
}





%%%%%%%%%%%%%%%%%%%%%%%%%%%%%%%%%%%%%%%%%%%%%%%%%%%%%%%%%%%%%%%%%%%%%%%%%%%%%%%%
%% Document
%%%%%%%%%%%%%%%%%%%%%%%%%%%%%%%%%%%%%%%%%%%%%%%%%%%%%%%%%%%%%%%%%%%%%%%%%%%%%%%%

\begin{document}

\maketitle

%%%%%%%%%%%%%%%%%%%%%%%%%%%%%%%%%%%%%%%%%%%%%%%%%%%%%%%%%%%%%%%%%%%%%%%%%%%%%%%%
%% Abstract
%%%%%%%%%%%%%%%%%%%%%%%%%%%%%%%%%%%%%%%%%%%%%%%%%%%%%%%%%%%%%%%%%%%%%%%%%%%%%%%%

\begin{abstract}
Professional bike fitting services are expensive and inconvenient, creating barriers for everyday cyclists. We propose a machine-learning pipeline that takes a short video of a rider cycling and outputs recommendations to improve their bike fit and posture. A key challenge is that, in realistic recordings, the rider is not guaranteed to be perfectly perpendicular to the camera. Standard 2D pose estimation methods therefore produce joint angles that are distorted by perspective. To address this, we introduce a pipeline that estimates the bicycle's yaw angle relative to the camera using a custom angle-prediction model, and then uses this estimate to account for perspective effects when interpreting 2D joint angles. Our approach provides near real-time feedback from any camera-enabled device with reasonable computational overhead.
\end{abstract}

%%%%%%%%%%%%%%%%%%%%%%%%%%%%%%%%%%%%%%%%%%%%%%%%%%%%%%%%%%%%%%%%%%%%%%%%%%%%%%%%
%% Introduction
%%%%%%%%%%%%%%%%%%%%%%%%%%%%%%%%%%%%%%%%%%%%%%%%%%%%%%%%%%%%%%%%%%%%%%%%%%%%%%%%

\section{Introduction and Motivation}
\label{sec:intro}


\subsection{Introduction to the Problem}
\label{subsec:intro}

Competitive cyclists often rely on professional bike fitting tools to improve performance and comfort. However, for everyday commuters, or even for amateur level competitive cyclists, these services are typically expensive and inconvenient, creating a high barrier to entry. Cheaper methods, such as the Hamley method or Holmes method of bike fitting provide useful heuristics such as inseam length and max knee flexion to set saddle heights\cite{peveler2011effects}, but these may result in suboptimal adjustments. A poorly fitted bike may result in overuse injuries or reduced economy \cite{peveler2008effects}.

In this project, we propose a machine-learning pipeline that takes a short video of a rider cycling and outputs recommendations to improve their bike fit and posture. A key challenge is that, in realistic recordings, the rider is not guaranteed to be perfectly perpendicular to the camera. Standard 2D pose estimation methods therefore produce joint angles that are distorted by perspective. To address this, we introduce a pipeline that estimates the bicycle's yaw angle relative to the camera using a custom angle-prediction model, and then uses this estimate to account for perspective effects when interpreting 2D joint angles.



% Include a few very relevant related works and how your work relates to those, expanding on the previous section. We do not expect you to cover all previous works.

\subsection{Related Works}
\label{subsec:related_works}


\subsubsection{Commercial bike-fit applications}

The use of software (including machine-learning-based tools) for bike fitting is well established. For example, Bike Fast Fit Elite\footnote{\url{https://apps.apple.com/cn/app/bike-fast-fit-elite/id1145619812?l=en-GB}} uses measured joint angles to recommend adjustments. A common limitation of these tools is that they assume the rider is filmed while cycling on a stationary setup (e.g., a wheel on trainer). This requirement reduces accessibility for cyclists who do not own such equipment.


\begin{samepage}
We surveyed several popular mobile applications in this category. On Android, apps with publicly visible downloads include:
\begin{itemize}
    \item PedalPro -- Fast Bike Fit: 50k+ downloads
    \item Apiir Bike Fit: 10k+ downloads
    \item Meu Bikefit+: 10+ downloads
\end{itemize}

From our review, none of these applications support bike fitting without a trainer. The closest workaround we found was suggested by Bike Fast Fit in response to the question: ``Not having a trainer, could I just lean against the wall and pedal in the reverse direction?'' They replied: ``That's a creative solution and it should work for a rough estimate. Pedaling under load changes some angles, so reverse pedaling won't be as accurate.''\cite{bikefastfit_help} While this may produce a rough estimate, it still requires the rider to remain upright while pedalling backwards, which can be unsafe or impractical in many setups.
\end{samepage}

\subsubsection{Sports Science Research}

There exists a literature on optimal bicycle mechanics and fit. Swart \cite{swart2019cycling} provides a survey of studies analyzing optimal saddle height among cyclists across different skill levels, measuring angles by various means. Peveler and Green examine the effects of seat height on anaerobic power output \cite{peveler2008effects,peveler2011effects}, specifically between the seat heights resulting from the Hamley and Holmes method of seat height calibrations. 

Other studies such as those conducted by Holiday, Fisher, Theo, and Swart \cite{holliday2017static} have attempted to estimate the differences between mechanics derived from static bike fits versus on-road realized mechanics. They reference methods from Bini, Hume, and Croft in utilizing reflective markers to measure the maximum knee flexion of a rider in a dynamic setting\cite{bini2011effects}. 
\subsection{Overview of the Project}
\label{subsec:overview}
We aim to provide a robust recommendation framework for seat post height adjustments from a 2D video of a human rider on a bicycle. In the first step, we clip the video to include only our target bicycle and rider. Next, we detect human joints of the rider and bicycle wheels. From these segmentations, we estimate the yaw of the bicycle using a custom model and use that measurement to inform our estimate of human joint angles. These upper joint angles (knee, hip, elbow) are overlaid into the drivetrain cycle, calculated from the relative ankle locations, providing the necessary data to provide an informed saddle adjustment. Further optional additions include data pipeline modification to improve the crankshaft angle detection. Additionally, we deploy our model on a website such that users can upload their own videos at \hyperlink{https://bikefitting.vercel.app/}{https://bikefitting.vercel.app/}.

%%%%%%%%%%%%%%%%%%%%%%%%%%%%%%%%%%%%%%%%%%%%%%%%%%%%%%%%%%%%%%%%%%%%%%%%%%%%%%%%
%% Methodology
%%%%%%%%%%%%%%%%%%%%%%%%%%%%%%%%%%%%%%%%%%%%%%%%%%%%%%%%%%%%%%%%%%%%%%%%%%%%%%%%

\section{Methodology}
\label{sec:methodology}

% Explain the baseline architecture you used to build your algorithm on. You may reproduce figures from the original papers.

\subsection{Baseline Algorithm}
\label{subsec:baseline_algo}

YOLOv8-pose predicts a 2D human skeleton by estimating joint positions in each frame. From these keypoints, joint angles can be computed using standard vector geometry. A simple baseline therefore processes a video of a rider cycling, computes joint angles frame by frame and summaries each joint's motion using statistics such as the maximum and minimum observed angles.

However, this baseline implicitly assumes that the rider is viewed from a perfectly side on perspective. In practice videos recorded without a bike trainer will not be perpendicular to the camera which introduces perspective distortion in the 2D keypoints. As a result, the angles computed from 2D projections may not reflect the rider's true angles, motivating the need for perspective aware corrections.

A fit may be achieved by considering only maximum and minimum joint flexion of the lower limbs along the drive cycle\cite{peveler2011effects}. However, the bicycle frame often will block the view of YOLO's ankle point at maximum position in the drive cycle, and thus our algorithm attempts to augment blocked measurements by approximating the hidden information by using a gaussian process. 

\subsection{How we obtained the ground-truth bike angle}
To measure the bicycle’s true yaw angle, we mounted a BNO055 9-axis IMU module directly onto the bike frame. The sensor provides absolute orientation that used an accelerometer, gyroscope and magnetometer readings, allowing it to remain stable over time (i.e., it does not suffer from drift like a gyroscope only estimate) by referencing the Earth’s magnetic field.

During data collection, the IMU was connected via a long cable to a battery-powered Raspberry Pi, which was carried alongside the rider (either held by the rider or secured to the rear of the bicycle). The Raspberry Pi logged timestamped orientation measurements throughout each recording.

To synchronise the sensor log with the video stream, held up a phone with the accurate time to the camera start of each recording. This provided a reference timestamp (to millisecond resolution) that allowed the alignment between video frames and IMU measurements during post-processing.


% \textbf{TODO: Do this section, the paper is kind of complicated in what it does}

% Explain what you implemented to improve over the baseline. You may include figures to explain the idea and logic. Focus on the ideas and not the implementation.
\subsection{Algorithm Improvements}
\label{subsec:algo_improvements}
We considered multiple approaches to reduce perspective induced error in joint-angle estimates. One option was to use \textit{MediaPipe Pose} which outputs body landmarks in both normalized image coordinates and estimated 3D world coordinates. \cite{mediapipe_pose_landmarker}
In preliminary experiments on side-on cycling videos, we found that these 3D estimates were not sufficiently reliable for our use case, particularly under self-occlusion (e.g., when the side of the body facing away from the camera is partially hidden).

We therefore adopt an alternative strategy: we estimate the bicycle's yaw angle relative to the camera using a custom-trained angle prediction model. This yaw estimate is then used to interpret and filter 2D pose measurements in a way that is more robust to non-perpendicular camera viewpoints.

To approximate the location of hidden ankle joints, we remove data points where YOLOv8 confidence is less than some threshold $t$ and fit a periodic gaussian process to the data. 
\documentclass[11pt,a4paper]{article}

\usepackage[utf8]{inputenc}
\usepackage[T1]{fontenc}
\usepackage{amsmath,amssymb}
\usepackage{graphicx}
\usepackage{booktabs}
\usepackage{hyperref}
\usepackage{enumitem}
\usepackage{geometry}
\usepackage{xcolor}
\usepackage{listings}
\usepackage{algorithm}
\usepackage{algpseudocode}

\geometry{margin=2.5cm}

\definecolor{codegreen}{rgb}{0,0.6,0}
\definecolor{codegray}{rgb}{0.5,0.5,0.5}
\definecolor{codepurple}{rgb}{0.58,0,0.82}

\lstset{
    basicstyle=\ttfamily\small,
    keywordstyle=\color{codepurple},
    commentstyle=\color{codegreen},
    stringstyle=\color{codegray},
    breaklines=true,
    frame=single
}

\title{BikeFit AI: Data Pipeline Architecture}
\author{Technical Documentation}
\date{\today}

\begin{document}

\maketitle

\begin{abstract}
This document describes the complete data processing pipeline for the BikeFit AI web application. The system uses a combination of computer vision, Bayesian optimisation with Gaussian Processes, and biomechanical heuristics to analyse cycling videos and generate bike fit recommendations. The pipeline is designed for efficiency, processing only $\sim$30 intelligently selected frames rather than the entire video.
\end{abstract}

\tableofcontents
\newpage

%==============================================================================
\section{System Overview}
%==============================================================================

The BikeFit AI system consists of two main components:

\begin{enumerate}
    \item \textbf{Frontend}: A Next.js web application that handles video upload and displays results
    \item \textbf{Backend}: A Modal serverless GPU function that performs the video analysis
\end{enumerate}

The data flows through seven main stages:

\begin{enumerate}
    \item Video Upload and Transmission
    \item Frame Scanning and Validation
    \item Active Learning Frame Selection
    \item Pose Detection and Angle Extraction
    \item Perspective Distortion Correction
    \item Gaussian Process Curve Prediction
    \item Recommendation Generation
\end{enumerate}

%==============================================================================
\section{Stage 1: Video Upload and Transmission}
%==============================================================================

\subsection{Frontend Processing}

When a user drops or selects a video file, the frontend performs the following steps:

\begin{enumerate}
    \item \textbf{File Validation}: Check that the file is a valid video format (MP4, MOV, AVI) and does not exceed 200MB
    \item \textbf{Base64 Encoding}: Convert the video file to Base64 string for transmission
    \item \textbf{HTTP Request}: Send a POST request to the Modal serverless endpoint with the encoded video
\end{enumerate}

\subsection{Server-Sent Events (SSE)}

The backend streams progress updates back to the frontend using Server-Sent Events. This allows real-time progress indication without polling:

\begin{lstlisting}[language=Python]
yield f"data: {json.dumps({
    'type': 'progress',
    'message': 'Analyzing video...',
    'percent': 50
})}\n\n"
\end{lstlisting}

\subsection{Backend Initialisation}

Upon receiving the video, the Modal backend:

\begin{enumerate}
    \item Decodes the Base64 video data
    \item Writes it to a temporary file on the serverless container
    \item Loads the required AI models (cached between requests)
\end{enumerate}

%==============================================================================
\section{Stage 2: Frame Scanning and Validation}
%==============================================================================

Not all frames in a cycling video are suitable for analysis. The system must identify frames where:
\begin{itemize}
    \item A bicycle is clearly visible
    \item The bicycle is viewed from approximately side-on (profile view)
    \item The cyclist's body is not occluded
\end{itemize}

\subsection{Bike Segmentation}

The first step uses YOLOv8 segmentation to detect and isolate the bicycle:

\begin{algorithm}
\caption{Bike Segmentation}
\begin{algorithmic}[1]
\Require Frame image $I$
\Ensure Masked bike image $M_{224}$, success flag
\State $results \gets \text{YOLO}(I)$
\For{each detection in $results$}
    \If{class = bicycle (COCO class 1)}
        \State $mask \gets$ segmentation mask
        \State Dilate $mask$ with $15 \times 15$ kernel
        \State Compute bounding box and crop to square
        \State Apply mask to cropped region
        \State Resize to $224 \times 224$
        \State \Return $(M_{224}, \text{True})$
    \EndIf
\EndFor
\State \Return $(\text{zeros}(224, 224, 3), \text{False})$
\end{algorithmic}
\end{algorithm}

\subsection{Bike Angle Prediction}

A custom-trained ConvNeXt model predicts the bicycle's yaw angle relative to the camera:

\begin{itemize}
    \item \textbf{Input}: $224 \times 224$ masked bike image
    \item \textbf{Output}: 120-bin classification over $[0°, 360°)$
    \item \textbf{Angle Computation}: Circular mean of softmax probabilities
\end{itemize}

The angle prediction uses soft classification with circular mean to handle the wraparound at $0°/360°$:

\begin{equation}
    \theta = \arctan2\left(\sum_{i} p_i \sin(\theta_i), \sum_{i} p_i \cos(\theta_i)\right)
\end{equation}

where $p_i$ is the softmax probability for bin $i$ and $\theta_i$ is the centre angle of bin $i$.

\subsection{Side-View Gating}

Only frames with the bike at approximately $90°$ to the camera are retained:

\begin{equation}
    \text{valid} = 60° \leq |\theta| \leq 120°
\end{equation}

This gating ensures that:
\begin{itemize}
    \item Joint angles can be accurately measured in 2D
    \item Perspective distortion is minimised
    \item The cyclist's near-side limbs are visible
\end{itemize}

\subsection{Scan Parameters}

\begin{itemize}
    \item \textbf{Scan Rate}: 30 fps (subsampled from original video)
    \item \textbf{Maximum Duration}: 120 seconds
    \item \textbf{Minimum Valid Frames}: 10 (otherwise analysis fails)
\end{itemize}

%==============================================================================
\section{Stage 3: Active Learning Frame Selection}
%==============================================================================

Rather than processing all valid frames (which could be hundreds), the system uses Gaussian Process-based active learning to intelligently select approximately 30 frames that maximise information gain.

\subsection{Motivation}

Joint angles during cycling follow a periodic pattern corresponding to the pedal stroke. By modelling this pattern with a Gaussian Process, we can:

\begin{enumerate}
    \item Estimate the full angle curve from sparse observations
    \item Quantify uncertainty at unobserved time points
    \item Select new samples where uncertainty is highest
\end{enumerate}

\subsection{Gaussian Process Model}

For each joint angle (knee, hip, elbow), we maintain a separate GP:

\begin{equation}
    f(t) \sim \mathcal{GP}(0, k(t, t'))
\end{equation}

where $k(t, t')$ is the covariance function. We use an RBF (Radial Basis Function) kernel:

\begin{equation}
    k(t, t') = \sigma^2 \exp\left(-\frac{(t - t')^2}{2\ell^2}\right)
\end{equation}

The lengthscale $\ell$ is given a Gamma prior informed by typical pedalling cadence:

\begin{equation}
    \ell \sim \text{Gamma}(2.0, 2.0 / \ell_{\text{target}})
\end{equation}

where $\ell_{\text{target}} = 0.4s / \sigma_t$ (normalised by time standard deviation).

\subsection{Acquisition Strategy}

The system uses \textbf{joint uncertainty} acquisition, prioritising frames where multiple joint angles have high predictive variance:

\begin{algorithm}
\caption{Joint Uncertainty Acquisition}
\begin{algorithmic}[1]
\Require Knee GP $\mathcal{M}_k$, Hip GP $\mathcal{M}_h$, visited indices $V$
\Ensure Next frame index to sample
\State $C \gets \{0, 1, \ldots, N-1\} \setminus V$ \Comment{Candidate indices}
\State $\sigma^2_k \gets \text{Var}[\mathcal{M}_k(t_C)]$ \Comment{Knee variance at candidates}
\State $\sigma^2_h \gets \text{Var}[\mathcal{M}_h(t_C)]$ \Comment{Hip variance at candidates}
\State $\sigma^2_{\text{joint}} \gets 0.8 \cdot \sigma^2_k + 0.2 \cdot \sigma^2_h$ \Comment{Weighted combination}
\State Apply spatial suppression near wasted indices
\State \Return $\arg\max_c \sigma^2_{\text{joint}}[c]$
\end{algorithmic}
\end{algorithm}

The weighting (80\% knee, 20\% hip) reflects the greater importance of knee angle for saddle height determination.

\subsection{Spatial Suppression}

When a frame fails pose detection (``wasted'' sample), nearby frames are suppressed to avoid repeated failures in similar regions:

\begin{equation}
    \sigma^2_{\text{joint}}[c] \gets -1 \quad \text{if} \quad \min_{w \in W} |t_c - t_w| < 1.0s
\end{equation}

\subsection{Initialisation and Optimisation}

\begin{enumerate}
    \item \textbf{Initialisation}: 5 random frames are sampled to bootstrap the GP
    \item \textbf{Hyperparameter Optimisation}: Every 5 samples, GP hyperparameters are re-fitted using marginal likelihood maximisation
    \item \textbf{Termination}: After 30 samples (or when no candidates remain)
\end{enumerate}

%==============================================================================
\section{Stage 4: Pose Detection and Angle Extraction}
%==============================================================================

For each selected frame, human pose is detected using YOLOv8-pose.

\subsection{Keypoint Detection}

YOLOv8-pose detects 17 COCO keypoints:

\begin{table}[h]
\centering
\begin{tabular}{cl|cl}
\toprule
Index & Keypoint & Index & Keypoint \\
\midrule
0 & Nose & 9 & Left Wrist \\
1 & Left Eye & 10 & Right Wrist \\
2 & Right Eye & 11 & Left Hip \\
3 & Left Ear & 12 & Right Hip \\
4 & Right Ear & 13 & Left Knee \\
5 & Left Shoulder & 14 & Right Knee \\
6 & Right Shoulder & 15 & Left Ankle \\
7 & Left Elbow & 16 & Right Ankle \\
8 & Right Elbow & & \\
\bottomrule
\end{tabular}
\caption{COCO keypoint indices}
\end{table}

\subsection{Side Detection}

The system automatically determines which side of the cyclist is facing the camera by counting confident keypoint detections:

\begin{equation}
    \text{side} = \begin{cases}
        \text{right} & \text{if } |\{k : \text{conf}_k^R > 0.5\}| \geq |\{k : \text{conf}_k^L > 0.5\}| \\
        \text{left} & \text{otherwise}
    \end{cases}
\end{equation}

\subsection{Joint Angle Calculation}

Three key angles are computed using the detected keypoints:

\subsubsection{Knee Angle}

The angle at the knee joint, measured between hip, knee, and ankle:

\begin{equation}
    \theta_{\text{knee}} = \angle(\vec{v}_{\text{hip} \to \text{knee}}, \vec{v}_{\text{ankle} \to \text{knee}})
\end{equation}

\subsubsection{Hip Angle}

The angle at the hip joint, measured between shoulder, hip, and knee:

\begin{equation}
    \theta_{\text{hip}} = \angle(\vec{v}_{\text{shoulder} \to \text{hip}}, \vec{v}_{\text{knee} \to \text{hip}})
\end{equation}

\subsubsection{Elbow Angle}

The angle at the elbow joint, measured between shoulder, elbow, and wrist:

\begin{equation}
    \theta_{\text{elbow}} = \angle(\vec{v}_{\text{shoulder} \to \text{elbow}}, \vec{v}_{\text{wrist} \to \text{elbow}})
\end{equation}

\subsection{Angle Computation}

The angle between two vectors is computed using the dot product:

\begin{equation}
    \theta = \arccos\left(\frac{\vec{u} \cdot \vec{v}}{|\vec{u}||\vec{v}|}\right)
\end{equation}

Values are returned in degrees. If any required keypoint has confidence below 0.5, the angle is marked as invalid (NaN).

\subsection{Confidence Threshold}

Only keypoints with confidence $> 0.5$ are used for angle calculation. If required keypoints are missing, the frame is added to the ``wasted'' set and spatial suppression is applied.

\subsection{Perspective Distortion Correction}

When the bicycle is not perfectly perpendicular to the camera (i.e., yaw $\neq 90°$), the measured 2D joint angles are subject to perspective distortion. The angles appear compressed due to foreshortening.

\subsubsection{Correction Factor}

The correction factor is derived from the camera geometry. For a bike at yaw angle $\theta$, the deviation from side-view is:

\begin{equation}
    \delta = |\theta - 90°|
\end{equation}

The measured angle is related to the true angle by the cosine of the deviation:

\begin{equation}
    \theta_{\text{measured}} \approx \theta_{\text{true}} \cdot \cos(\delta)
\end{equation}

Therefore, the correction factor is:

\begin{equation}
    f_{\text{correction}} = \frac{1}{\cos(\delta)}
\end{equation}

And the corrected angle is:

\begin{equation}
    \theta_{\text{corrected}} = \theta_{\text{measured}} \times f_{\text{correction}}
\end{equation}

\subsubsection{Example}

For a bicycle at yaw angle $80°$ (i.e., $10°$ off from perfect side-view):

\begin{itemize}
    \item Deviation: $\delta = |80° - 90°| = 10°$
    \item Correction factor: $f = 1 / \cos(10°) \approx 1.015$
    \item If measured knee angle is $140°$, corrected angle is $140° \times 1.015 \approx 142°$
\end{itemize}

\subsubsection{Implementation}

The correction is applied during the \texttt{\_process\_sample} method:

\begin{lstlisting}[language=Python]
deviation_deg = abs(abs(yaw) - 90)
deviation_rad = np.radians(deviation_deg)
correction_factor = 1.0 / np.cos(deviation_rad)

knee_angle = measured_knee * correction_factor
hip_angle = measured_hip * correction_factor
elbow_angle = measured_elbow * correction_factor
\end{lstlisting}

This correction is only applied for frames within the $\pm 30°$ gating range (i.e., $60° - 120°$), where the correction factor ranges from $1.0$ (at $90°$) to approximately $1.15$ (at $60°$ or $120°$).

%==============================================================================
\section{Stage 5: Gaussian Process Curve Prediction}
%==============================================================================

After sampling is complete, the GP models are used to predict the full joint angle curves.

\subsection{Posterior Prediction}

Given observations $\mathbf{y}$ at times $\mathbf{t}$, the GP posterior at all valid frame times $\mathbf{t}_*$ is:

\begin{align}
    \mathbb{E}[f(\mathbf{t}_*)] &= K(\mathbf{t}_*, \mathbf{t}) [K(\mathbf{t}, \mathbf{t}) + \sigma_n^2 I]^{-1} \mathbf{y} \\
    \text{Var}[f(\mathbf{t}_*)] &= K(\mathbf{t}_*, \mathbf{t}_*) - K(\mathbf{t}_*, \mathbf{t}) [K(\mathbf{t}, \mathbf{t}) + \sigma_n^2 I]^{-1} K(\mathbf{t}, \mathbf{t}_*)
\end{align}

\subsection{Key Metrics Extraction}

From the predicted curves, the following metrics are extracted:

\begin{table}[h]
\centering
\begin{tabular}{lll}
\toprule
Metric & Computation & Biomechanical Meaning \\
\midrule
Max Knee Extension & $\max_t \hat{f}_{\text{knee}}(t)$ & Leg straightness at bottom of stroke \\
Min Knee Flexion & $\min_t \hat{f}_{\text{knee}}(t)$ & Knee bend at top of stroke \\
Min Hip Angle & $\min_t \hat{f}_{\text{hip}}(t)$ & Hip closure at top of stroke \\
Avg Elbow Angle & $\frac{1}{N}\sum_t \hat{f}_{\text{elbow}}(t)$ & Arm bend on handlebars \\
\bottomrule
\end{tabular}
\caption{Key metrics extracted from GP predictions}
\end{table}

%==============================================================================
\section{Stage 6: Recommendation Generation}
%==============================================================================

The extracted metrics are converted to actionable bike fit recommendations using biomechanical heuristics.

\subsection{Saddle Height}

Based on maximum knee extension angle:

\begin{itemize}
    \item \textbf{Target Range}: $140° - 150°$
    \item \textbf{If $\theta < 140°$}: Saddle too low. Raise by $(\theta_{\text{target}} - \theta) \times 2$ mm
    \item \textbf{If $\theta > 150°$}: Saddle too high (overextension risk). Lower by $(\theta - \theta_{\text{target}}) \times 2$ mm
\end{itemize}

The $2$ mm/degree rule is a common bike fitting heuristic.

\subsection{Saddle Fore/Aft Position}

Based on minimum knee flexion angle:

\begin{itemize}
    \item \textbf{Target}: $> 70°$
    \item \textbf{If $\theta < 70°$}: Knee too closed at top of stroke. Move saddle backward 5-10 mm
\end{itemize}

\subsection{Crank Length}

Based on hip and knee clearance at top of stroke:

\begin{itemize}
    \item \textbf{Target}: Hip $> 48°$ AND Knee $> 68°$
    \item \textbf{If impingement detected}: Consider shorter cranks (-5 mm)
\end{itemize}

Shorter cranks open the hip angle and reduce knee flexion at the top of the stroke.

\subsection{Stem/Reach}

Based on average elbow angle:

\begin{itemize}
    \item \textbf{Target Range}: $150° - 160°$
    \item \textbf{If $\theta > 160°$}: Arms too straight. Shorten stem by $\max(10, \frac{\theta - 160}{5} \times 10)$ mm
    \item \textbf{If $\theta < 150°$}: Arms too bent. Lengthen stem by $\max(10, \frac{150 - \theta}{5} \times 10)$ mm
\end{itemize}

\subsection{Output Format}

Recommendations are returned as a structured JSON object:

\begin{lstlisting}[language=Python]
{
    "saddle_height": {
        "status": "low",
        "action": "raise",
        "adjustment_mm": 15,
        "details": "Knee ext 135 deg below optimal. Raise ~15mm."
    },
    "saddle_fore_aft": { ... },
    "crank_length": { ... },
    "cockpit": { ... },
    "summary": ["Raise saddle ~15mm", ...],
    "metrics": {
        "knee_max_extension": 135.2,
        "knee_min_flexion": 72.5,
        "min_hip_angle": 52.3,
        "avg_elbow_angle": 158.1
    }
}
\end{lstlisting}

%==============================================================================
\section{Performance Characteristics}
%==============================================================================

\subsection{Timing}

\begin{table}[h]
\centering
\begin{tabular}{lr}
\toprule
Stage & Typical Duration \\
\midrule
Video Upload (10MB) & 2-5 s \\
Model Loading (cached) & 1-2 s \\
Frame Scanning (1000 frames) & 5-10 s \\
Active Learning (30 samples) & 3-5 s \\
GP Prediction & $<$ 1 s \\
Recommendation Generation & $<$ 0.1 s \\
\midrule
\textbf{Total} & \textbf{15-30 s} \\
\bottomrule
\end{tabular}
\caption{Typical processing times on NVIDIA T4 GPU}
\end{table}

\subsection{Efficiency}

The active learning approach provides significant efficiency gains:

\begin{itemize}
    \item \textbf{Traditional approach}: Process all frames (potentially 1000+)
    \item \textbf{Our approach}: Process only $\sim$30 strategically selected frames
    \item \textbf{Speedup}: $\sim$30x reduction in pose detection calls
\end{itemize}

%==============================================================================
\section{Error Handling}
%==============================================================================

\subsection{Insufficient Valid Frames}

If fewer than 10 valid side-view frames are found:

\begin{lstlisting}
{"stats": {"frames_processed": 0, "error": "Not enough side-view frames"}}
\end{lstlisting}

\textbf{Causes}: Video not from side view, no bicycle visible, poor lighting.

\subsection{Pose Detection Failures}

If pose detection fails on a selected frame:
\begin{itemize}
    \item Frame is added to ``wasted'' set
    \item Spatial suppression prevents nearby re-sampling
    \item GP continues with remaining observations
\end{itemize}

\subsection{Insufficient Samples}

If fewer than 10 successful pose detections are obtained, recommendations may be unreliable. The system proceeds but confidence is reduced.

%==============================================================================
\section{Technology Stack}
%==============================================================================

\begin{table}[h]
\centering
\begin{tabular}{ll}
\toprule
Component & Technology \\
\midrule
Frontend & Next.js 15, React, TypeScript, Tailwind CSS \\
Backend & Modal (serverless), Python 3.11, FastAPI \\
GPU & NVIDIA T4 (16GB) \\
Bike Segmentation & YOLOv8n-seg \\
Bike Angle Prediction & ConvNeXt-Tiny (custom trained) \\
Pose Detection & YOLOv8m-pose \\
Gaussian Processes & BoTorch, GPyTorch \\
\bottomrule
\end{tabular}
\caption{Technology stack}
\end{table}

%==============================================================================
\section{Conclusion}
%==============================================================================

The BikeFit AI data pipeline combines modern deep learning with classical Bayesian optimisation to efficiently analyse cycling videos. Key innovations include:

\begin{enumerate}
    \item \textbf{Intelligent Frame Selection}: Using Gaussian Process-based active learning to sample only the most informative frames
    \item \textbf{Multi-Stage Filtering}: Bike detection, angle validation, and pose confidence thresholds ensure robust analysis
    \item \textbf{Biomechanical Heuristics}: Converting raw joint angles to actionable adjustment recommendations
\end{enumerate}

The result is a system that can provide professional-level bike fit recommendations in under 30 seconds from a simple phone video.

\end{document}




% Explain which framework and optimizers you use, how you implemented the training logic.
\subsection{Training Procedures}
\label{subsec:training_procedures}
% Todo, maybe cite some stuff here like exponentailly decaying lr, pytorch, etc.
\begin{figure}[H]
    \centering

    \begin{subfigure}{0.32\textwidth}
        \centering
        \adjustbox{width=\linewidth,height=4.2cm,center,clip}{%
            \includegraphics{Chapters/methodology/CollectingData1.jpeg}
        }
    \end{subfigure}
    \hfill
    \begin{subfigure}{0.32\textwidth}
        \centering
        \adjustbox{width=\linewidth,height=4.2cm,center,clip}{%
            \includegraphics{Chapters/methodology/CollectingData3.jpeg}
        }
    \end{subfigure}
    \hfill
    \begin{subfigure}{0.32\textwidth}
        \centering
        \adjustbox{width=\linewidth,height=4.2cm,center,clip}{%
            \includegraphics{Chapters/methodology/CollectingData2.jpeg}
        }
    \end{subfigure}

    \caption{Data collection setup used for training the bike yaw angle model.}
    \label{fig:data_collection}
\end{figure}


%%%%%%%%%%%%%%%%%%%%%%%%%%%%%%%%%%%%%%%%%%%%%%%%%%%%%%%%%%%%%%%%%%%%%%%%%%%%%%%%
%% Experiments
%%%%%%%%%%%%%%%%%%%%%%%%%%%%%%%%%%%%%%%%%%%%%%%%%%%%%%%%%%%%%%%%%%%%%%%%%%%%%%%%

\section{Experiments and Evaluation}
\label{sec:experiment}

% Explain the datasets utilized: what they contain, why they are utilized, assumptions, limitations, possible extensions.
\subsection{Datasets}
\label{subsec:datasets}

\subsubsection{Seatpost Height}

\subsubsection{Crankshaft Position Detection}
We derived 

For the crank angle augmentation method, we take high-confidence YOLOv8 points as approximate truth and fit our GP according to these data. We used videos of professional cyclists to tune the parameters of our model \cite{twoRiders, pogi, side1}. 

\begin{figure}[H]
    \centering
    \includegraphics[width=0.5\linewidth]{Chapters/experiment/crank_variance_side1.mp4.png}
    \caption{Crank Angle GP}
    \label{fig:crank-gp}
\end{figure}

\begin{figure}[H]
    \centering
    \includegraphics[width=0.5\linewidth]{tadej1.png}
    \caption{Tadej Pogačar in the 2024 Tour De France}
    \label{fig:placeholder}
\end{figure}

% Explain the training and testing results with graphs and elaborating on why they make sense, what could be improved.
\FloatBarrier
\subsection{Training and Testing Results}
\label{subsec:results}


\begin{figure}[H]
    \centering
    
    \begin{subfigure}[c]{0.38\textwidth}
        \centering
        \includegraphics[width=\linewidth]{Chapters/experiment/IMG_3803_frame_000486.jpg}
        \caption{Sample validation frame}
        \label{fig:validation-frame}
    \end{subfigure}
    \hfill
    \begin{subfigure}[c]{0.58\textwidth}
        \centering
        \includegraphics[width=\linewidth]{Chapters/experiment/validation_line_graph.png}
        \caption{True vs predicted angle. The dip is due to wrap around.}
        \label{fig:validation-graph}
    \end{subfigure}

    \caption{Bike angle prediction validation on unseen video data.}
    \label{fig:validation-results}
\end{figure}

\FloatBarrier

\begin{table}[H]
    \centering
    \caption{Validation metrics}
    \label{tab:validation-metrics}
    \begin{tabular}{lcc}
        \toprule
        \textbf{Metric} & \textbf{Raw} & \textbf{Smoothed (30 frame window)} \\
        \midrule
        Mean Absolute Error & 10.67° & 8.87° \\
        Median Absolute Error & 8.60° & 8.26° \\
        90th Percentile Error & 23.09° & 17.15° \\
        \bottomrule
    \end{tabular}
\end{table}

\subsubsection{Crankshaft Identification Method}


%%%%%%%%%%%%%%%%%%%%%%%%%%%%%%%%%%%%%%%%%%%%%%%%%%%%%%%%%%%%%%%%%%%%%%%%%%%%%%%%
%% Conclusion
%%%%%%%%%%%%%%%%%%%%%%%%%%%%%%%%%%%%%%%%%%%%%%%%%%%%%%%%%%%%%%%%%%%%%%%%%%%%%%%%

\section{Conclusions and Future Directions}
\label{sec:conclusion}

% Summarize what the project was about and the main conclusions.

\subsection{Summary}
\label{subsec:summary}
We find that video-based bike-fitting methods can effectively give a recommendation to raise or lower the seat post using a Holmes seat adjustment method. Further, our model provides feedback in near real time from any camera-enabled device with reasonable computational overhead. While we do not recommend our model to be used for racing or aerodynamic optimization, the product can improve the fit for a user's bike at effectively no cost. 
% Explain the limitations of your technique. You may want to refer to previous sections or show figures on the limitations.

\subsection{Discussion of Limitations}
\label{subsec:limitation}
Most state-of-the-art bike-fitting applications take exact measurements of ankle flexion to understand the entire drive dynamic. Due to the difficulty in measuring the ankle position using our technique, we must use a heuristic derived from crank position, which may result in imperfect measurements. Additionally, we do not require data quality checks from user inputs, such that poor video can result in limited recommendation confidence. 


\input{Chapters/conclusion/future}

%%%%%%%%%%%%%%%%%%%%%%%%%%%%%%%%%%%%%%%%%%%%%%%%%%%%%%%%%%%%%%%%%%%%%%%%%%%%%%%%
%% Acknowledgments (optional, only visible in final version)
%%%%%%%%%%%%%%%%%%%%%%%%%%%%%%%%%%%%%%%%%%%%%%%%%%%%%%%%%%%%%%%%%%%%%%%%%%%%%%%%

% \begin{ack}
% We thank...
% \end{ack}

%%%%%%%%%%%%%%%%%%%%%%%%%%%%%%%%%%%%%%%%%%%%%%%%%%%%%%%%%%%%%%%%%%%%%%%%%%%%%%%%
%% References
%%%%%%%%%%%%%%%%%%%%%%%%%%%%%%%%%%%%%%%%%%%%%%%%%%%%%%%%%%%%%%%%%%%%%%%%%%%%%%%%

\bibliographystyle{plainnat}
\bibliography{thesis}

%%%%%%%%%%%%%%%%%%%%%%%%%%%%%%%%%%%%%%%%%%%%%%%%%%%%%%%%%%%%%%%%%%%%%%%%%%%%%%%%
%% Appendix
%%%%%%%%%%%%%%%%%%%%%%%%%%%%%%%%%%%%%%%%%%%%%%%%%%%%%%%%%%%%%%%%%%%%%%%%%%%%%%%%

\appendix

\section{Additional Figures}
\label{appendix:figures}
\input{Chapters/appendix/figures}

\section{Contributions}
\label{appendix:contribution}
\subsection{Author Contributions}
\label{subsec:contribution}

\subsubsection{Luca Vendruscolo}

\begin{itemize}
    \item \textbf{Joint angle detection}: Implemented the YOLOv8 pose estimation model to get joint angles.
    \item \textbf{Data collection}: Made a portable Raspberry Pi setup with a gyroscope sensor to get training data.
    \item \textbf{Model training}: Trained a ConvNeXt model to find the angle of the bike.
    \item \textbf{Web application}: Created the inital framework for the website's front and back ends.
    \item \textbf{Video processing}: Added serverless GPU processing to the backend for the website.
    \item \textbf{Documentation}: Helped with writing documentation.
\end{itemize}

%This is an inital draft 

\subsubsection{Kyle Fram}

\begin{itemize}
    \item \textbf{Data collection}: Helped get data for the bike angle model.
    \item \textbf{Pose estimation improvements}: Added ankle angle estimation.
    \item \textbf{Domain knowledge}: Looked at cycling biomechanics research to suggest adjustments to the bike.
    \item \textbf{Documentation}: Helped with writing documentation.
\end{itemize}

\subsubsection{Fivos Papanikolaou}

\begin{itemize}
    \item \textbf{Data collection}: Helped get data for the bike angle model.
    \item \textbf{Bayesian optimisation}: Implemented Bayesian optimisation for estimating angles.
    \item \textbf{Fit recommendations}: Used angles from the models to create the recommendation logic.
    \item \textbf{Documentation}: Helped with writing documentation.
\end{itemize}

\subsection{Experimentation}

\paragraph{EMPTY} 
\begin{enumerate}
    \item 
\end{enumerate}


\end{document}
